\documentclass[11pt]{article}
\usepackage[a4paper,margin=1mm]{geometry}
\usepackage[style=ieee,url=false,doi=false,maxbibnames=99,sorting=ydnt,dashed=false]{biblatex}
\bibliography{papers}
% https://en.wikibooks.org/wiki/LaTeX/Colors
\def\theme{OliveGreen}

\usepackage{simplecv}

\begin{document}

\headinginline{Tomasz Nowak}
	{
    Email: \email{tomasz.nowak@tonowak.com} \\
    LinkedIn: \linkedin{tomasz-nowak-tonowak} \\
}{
    GitHub: \github{tonowak} \\
    GitLab: \gitlab{tonowak} \\
}% {PA_zdjecie_twarzy.jpg}

\section{Experience}
\outerlist{
	\entry{\textbf{Jane Street -- Software Developer Internship} \hfill London, 2022 and 2023}
	\innerlist {
		\entry{\textbf{2023, first project:}
				worked with a trader/researcher to implement
				a tool for analyzing and calculating
				some historical data of securities
				for determining risks and benefits of holding a bond.
		}
		\entry{\textbf{2023, second project:}
				made OCaml work with gdb/lldb;
				extended the OCaml compiler with debug
				info of variable types
				and implemented support for OCaml
				and its variant types in lldb
				(a part of the llvm project).
		}
		\entry{\textbf{2022, first project:}
				extended a company product for their clients
				to support multiple types of securities.
		}
		\entry{\textbf{2022, second project:}
				designed and implemented an asynchronous system for
				retrieving and sending data across multiple
				sources, with a focus on reconnections,
				error handling and correctness.
		}
	}
}

\section{Education}
\outerlist{
	\entrybig
	{University of Warsaw}{}
	{\textbf{Bachelor's degree} in Computer Science, \textbf{GPA 4.95 / 5.00.}}{2020\textendash 2023}
	\innerlist{
		\entry{One of the \textbf{2 students out of around 160}
			to receive an ,,Exceptional 5!'' final grade.}
		\entry{\link{https://arxiv.org/abs/2308.14623}{Thesis}:
		``Accelerating package expansion in Rust through development of a semantic versioning tool''.

		\textbf{Wrote a blog post} (link soon) describing the important results of the thesis.

		My code in this project is now used by companies like \textbf{Microsoft, Amazon, Mozilla and Adobe},
		and by some of the biggest Rust libraries.}
	}

	\entrybig
	{University of Warsaw}{}
	{\textbf{Master's degree} in Computer Science, work in progress.}{2023\textendash 2025}
	\innerlist{
		\entry{\textbf{Teaching:} I'm conducting classes in \textit{Text algorithms}
		(one of the additional classes that bachelor's and master's students have to choose from)
		while still being a student.}
	}
}

\section{Competitions (sports programming)}
\denseouterlist{
	\entry{ICPC World Finals: \textbf{finalist of ICPC WF 2022} (which will happen in November 2023).}
	\entry{ICPC Central Europe Regional Contest: \textbf{golden medalist of CERC 2022, medalist of CERC 2021 and 2020}.}
	\entry{Central European Olympiad in Informatics: \textbf{winner of CEOI 2020}.}
	\entry{International Olympiad in Informatics: \textbf{medalist of IOI 2020}.}
	\entry{Polish Olympiad in Informatics: \textbf{winner of POI 2020}, medalist of POI 2019, finalist of POI 2018.}
	\entry{Polish Olympiad in Mathematics: finalist of POM 2020 and POM 2019.}
	% \entry{finalist of PA 2020 and PA 2019 (Algorithmic Engagements / Potyczki Algorytmiczne, open competition in Poland) \hfill 2018\textendash 2020}
}

\section{Organization of competitions}
\denseouterlist{
	\entry{\textbf{Vicechair of the Jury} of the Polish Olympiad in Informatics (2023 -- 2025).}
	\entry{Member of the Jury of the Polish Olympiad in Informatics and Junior Olympiad in Informatics, member of Polish Olympiad in Informatics' Task Commision (2020 -- Current).}
	\entry{Team leader/deputy leader of the Polish national team in IOI 2021, CEOI 2021, BOI 2021, BOI 2022, BOI 2023.}
	\entry{Each year organizing one or two algorithmic camps for high school students to prepare them for competitions.}
}

\sidebyside{
	\section{Languages}
	English, Polish, French
}{
	\section{Programming languages}
	Rust, C++, OCaml, Python, C, Java, Haskell
}

\end{document}
